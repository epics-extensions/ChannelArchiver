\chapter{ArchiveDaemon}
\begin{figure}[htb]
\begin{center}
\InsertImage{width=0.8\textwidth}{daemon}
\end{center}
\caption{\label{fig:daemon}Archive Daemon, refer to text.}
\end{figure}

\noindent The ArchiveDaemon is a script that automatically starts,
monitors and restarts ArchiveEngines on the local host. It includes a
built-in web server, so by listing all the ArchiveEngines that are
meant to run on a host in the ArchiveDaemon's configuration file, one
can check the status of all these engines on a single web page as
shown in Fig.~\ref{fig:daemon}.

The daemon will attempt to start any ArchiveEngine that it does not
find running. In addition, the daemon can periodically stop and
restart ArchiveEngines in order to create e.g.\ daily sub-archives.
Furthermore, it adds each sub-archive to a configuration file for the
ArchiveIndexTool and runs the latter, so that all the sub-archives can
be accessed as if they were one big archive.

Before attempting to use the ArchiveDaemon, one should be familiar
with the configuration of the ArchiveEngine (sec.\ \ref{sec:engine}),
and how to start and stop it. Furthermore, one needs to be familiar
with the ArchiveIndexTool (sec.\ \ref{sec:indextool}).

\subsection{Configuration}
The ArchiveDaemon expects to find a configuration file called
``ArchiveDaemon.xml'' in the directory where it is started.  That
configuration file, an example of which can be found in listing
\ref{lst:daemonconfigex}, needs to follow the DTD from listing
\ref{lst:daemonconfigdtd}.

\lstinputlisting[float=htb,keywordstyle=\sffamily,caption={Example Archive Daemon Configuration},label=lst:daemonconfigex]{../ArchiveDaemon/ArchiveDaemon.xml}

\lstinputlisting[float=htb,keywordstyle=\sffamily,caption={XML DTD for
    the Archive Daemon Configuration},label=lst:daemonconfigdtd]{../ArchiveDaemon/ArchiveDaemon.dtd}

\noindent The configuration lists all the
ArchiveEngines that the daemon should manage on the local
computer. One ``engine'' element per ArchiveEngine specifies the
configuration of each engines. Specifically, the following tags are
allowed:

\subsubsection{desc}
This mandatory element is used for the ``-description'' option of the
Archive Engine, see section \ref{sec:enginedesc}.

\subsubsection{port}
This mandatory element determines the port number of the engine's HTTP
server, see section \ref{sec:engineport}.

\NOTE The ArchiveDaemon itself requires a TCP port number for its HTTP
server. The port numbers used by the ArchiveDaemon and all the Archive
Engines need to be different. You cannot use the same port number more
than once per computer.

\subsubsection{config}
This mandatory element must contain the full path to the configuration
file of the respective ArchiveEngine, see section \ref{sec:engineconfig}.

\subsubsection{daily}
This optional element configures the ArchiveDaemon to restart the
ArchiveEngine periodically. The element must contain a time in the
format ``HH:MM'' with 24-hour hours HH and minutes MM. One example
would be ``02:00'' for a restart at 2~am each morning.

\subsection{Operation}
The ArchiveDaemon is a perl script that is typically started like this:

\begin{lstlisting}[keywordstyle=\sffamily]
$ cd whereever_you_placed_ArchiveDaemon.xml
$ perl ArchiveDaemon.pl
Read ArchiveDaemon.xml, will disassociate from terminal
and from now on only respond via
          http://localhost:4610
You can also monitor the log file:
          ArchiveDaemon.log
\end{lstlisting}

\noindent One can use any web browser to connect to the daemon's HTTP server
under the URL shown in the above status message. Fig.~\ref{fig:daemon}
shows one example. The ArchiveDaemon offers a command line option for
selecting a specific TCP port.
Whenever running more than one ArchiveDaemon per computer, they
need to be started with different TCP port numbers. Furthermore, each
ArchiveEngine needs a different TCP port number.
\begin{lstlisting}[keywordstyle=\sffamily]
USAGE: ArchiveDaemon [-p port] 
       -p <port> : TCP port number for HTTPD
\end{lstlisting}

\noindent The ArchiveDaemon will create or use the following files in the
directory in which it was started:
\begin{itemize}
\item ArchiveDaemon.log\\
  The log file of the ArchiveDaemon.
\item indexconfig.xml\\
  This is a configuration file for the ArchiveIndexTool.  If the file
  already exists, the ArchiveDaemon will add every new sub-archive
  that it creates to the file. If the file does not exist, one will be
  created the next time a new sub-archive is started.
\item ArchiveIndexTool.log\\
  The log file of the last run of the ArchiveIndexTool.
\item master\_index\\
  The ArchiveIndexTool is run with indexconfig.xml to update
  this master index file.
\end{itemize}

\noindent The ArchiveDaemon configuration file must list the full path
names to the configuration files for the ArchiveEngines.
Within each of those directories, an ArchiveEngine is run and the
following files will be created:
\begin{itemize}
\item ArchiveEngine.log\\
  A log file for the ArchiveEngine running in that directory
\item archive\_active.lck\\
  Lock file of the ArchiveEngine
\item YYYY/MM\_DD/index \\
  A subdirectory for index and data files of the sub-archive.  If the
  ArchiveDaemon is configured to perform daily restarts, the format
  uses the year, month and day to build the path name.
\end{itemize}

\noindent To stop the ArchiveDaemon, access the ``/stop'' URL
of the daemon's HTTPD, e.g. ``http://localhost:4610/stop''.
