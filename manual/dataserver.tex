\section{Data Server} \label{sec:dataserver}
The archiver toolset includes a network data server.
By running this data server on a computer that has
physical access to your archived data, be it because the data resides
on a local disk or an NFS-mapped volume, other machines
on the network can get read-access to your data.

\begin{figure}[htb]
\begin{center}
\InsertImage{width=0.8\textwidth}{dataserver}
\end{center}
\caption{\label{fig:dataserver}Data Server, refer to text.}
\end{figure}

\noindent The data server is hosted by a web server, using the XML-RPC
protocol to serve the data. This means that software running on
disparate operating systems, running in different environments can
access your data over the Internet via a URL. As an example, your data
server might be a Linux machine on a subnet behind a firewall. After
you configure the firewall to pass HTTP requests, any Linux, Win32,
Macintosh computer both inside or outside of the firewall can access
the data from within perl, python or tcl scripts, programs written in
C, C++ or Java, actually pretty much any programming language. As
illustrated in fig.~\ref{fig:dataserver}, the client program sends its
requests to a web server, which forwards it to the data server that is
running as a CGI tool under the web server. The dataserver accesses
the relevant archives --- you determine which ones are available via a
configuration file --- and returns the data though the web server to
the client program.  You can configure access security via e.g.\ the
Apache web server configuration.

\NOTE The fact that the data server is hosted by a web server,
accessible via a URL, does \emph{not} imply that you can use any web browser
to retrieve data. You have to use the XML remote procedure call
protocol described in section \ref{sec:xmlprotocol} on page
\pageref{sec:xmlprotocol}. XML-RPC handles the network connections,
data type conversions, and XML-RPC libraries are available for most
programming languages.
We provide the Java Archive Client (see section \ref{sec:javaclient})
as a generic, interactive data client. 

\subsection{Installation} % ---------------------------------------------
After successful compilation in ChannelArchiver/XMLRPCServer, you will
have a program ``ArchiveDataServer''. You need to copy that binary as
``ArchiveDataServer.cgi'' into your web server's CGI directory and
assert that the web server can execute the ArchiveDataServer.
The ``.cgi'' extension is important, because otherwise your web server
might not recognize your CGI program as such.
What follows is an example setup for the Apache Web Server under Linux:
\begin{enumerate}
\item Check your Apache configuration file, which can often been found
  in ``/etc/httpd/conf/httpd.conf'', for the location of your cgi-bin
  directory (For Mandrake 10, check ``/etc/httpd/conf/commonhttpd.conf'').
  In my case, there was already a CGI directory configured as /var/www/cgi-bin.

\item Create a new CGI directory so that we can specifically configure
   it, e.g.\ /var/www/cgi-bin/xmlrpc, and then add the following to
   the Apache config file:
\begin{lstlisting}[keywordstyle=\sffamily]
<Directory /var/www/cgi-bin/xmlrpc>
   SetEnv LD_LIBRARY_PATH  /usr/local/lib:...
   SetEnv SERVERCONFIG /var/www/cgi-bin/ \
                   xmlrpc/serverconfig.xml
</Directory>
\end{lstlisting}
  The LD\_LIBRARY\_PATH needs to include all the directories that
  contain shared libraries which your ArchiveDataServer.cgi uses.
  In most cases, this includes the
  \begin{itemize}
  \item install location of the expat and XML-RPC
        libraries, often /usr/local/lib,
  \item ``lib'' subdirectories of EPICS base and EPICS extensions,
        something like /ade/epics/supTop/base/R3.14.4/lib/linux-x86:...
  \end{itemize}
  The SERVERCONFIG variable needs to point to your server configuration
  file, more about which next.
\end{enumerate}

\subsection{Setup}
You need to prepare an XML-formatted configuration file for the
ArchiveDataServer that follows the DTD from listing
\ref{lst:serverconfigdtd} (see section \ref{sec:dtdfiles} on DTD file
installation). Note that the ArchiveDataServer might not verify your
configuration file, so you are strongly encouraged to use a tool like
'xmllint' on Linux to check your configuration against the
DTD. Listing \ref{lst:serverconfigex} shows one example configuration
which lists two archives to be served. Client programs will internally
use the respective 'key' to access them.

\lstinputlisting[float=htb,keywordstyle=\sffamily,caption={XML DTD for the Data Server Configuration},label=lst:serverconfigdtd]{../XMLRPCServer/serverconfig.dtd}
\lstinputlisting[float=htb,keywordstyle=\sffamily,caption={Example Data Server Configuration},label=lst:serverconfigex]{../XMLRPCServer/serverconfig.xml}
